\documentclass{beamer}

\usepackage{verbatim}

\title{RFDev: Linux Kernel Driver for Routing Fabrics in AXIOM Beta Main Board}
\subtitle{GSoC 2020 - Phase 1 - Mid}
\author{Swaraj Hota}
\date

\begin{document}
{
	\setbeamertemplate{footline}{}
	\frame{\titlepage}
}

\section*{Contents}

\begin{frame}{Contents}
	\tableofcontents
\end{frame}

\section{Progress}

\subsection{Claim PIC addresses}

\begin{frame}[fragile]{Claim PIC addresses}
\begin{itemize}
\item Firstly, basic i2c driver code was added
\item The driver claims i2c addresses 0x40-0x4f, which PIC responds to
\item A device tree entry was added for this (inside the entry for i2cbus-2):
\begin{verbatim}
pic@40 {
	compatible = "apertus,pic-rf-interface";
	reg = <0x40 0x10 0x60 0x10>;
	#address-cells = <1>;
	#size-cells = <1>;
};
\end{verbatim}
\item Only 0x40 claimed by default, others claimed through ``dummy clients"
\end{itemize}
\end{frame}

\subsection{Create a Sysfs interface}

\begin{frame}{Create a Sysfs interface}
\begin{itemize}
\item A group of attributes for the driver found in /sys/devices/rfdev/
\item Currently, the group contains only ``idcode" attribute
\item Reading out idcode is the first successful test for communicating with
	the routing fabrics through the PIC
\item Other useful attributes can be added in this group (which are not provided
	by FPGA Manager)
\end{itemize}
\end{frame}

\subsection{Integrate FPGA Manager}

\begin{frame}[fragile]{Integrate FPGA Manager}
\begin{itemize}
\item A Linux Driver Framework which will allow us to easily upload firmware
	images into the MachXO2's SRAM
\item Once registered, provides a sysfs interface inside /sys/class/fpga\_manager/fpga\#/
\item ``firmware" attribute will allow us to do something like:
\begin{verbatim}
echo firmware.bin >
/sys/class/fpga_manager/fpga1/firmware
\end{verbatim}
\item API functions: write\_init, write and write\_complete, do the uploading
\item ``state" attribute will allow us to read the current state of the FPGA 
\end{itemize}
\end{frame}

\section{Upcoming}

\begin{frame}{Upcoming}
\begin{itemize}
\item To fill in the API functions by FPGA Manager with appropriate read/write
	commands to the PIC, and get a firmware uploaded successfully
\item To research and find a suitable way to implement a JTAG interface
\item JTAG interface will allow software like OpenOCD to debug the routing fabrics
	as well as upload/flash new firmware
\item To have a checksum/hash reading attribute that helps us to know which firmware
	is uploaded
\end{itemize}
\end{frame}

\section{Code}

\begin{frame}{Code}
Link to Github repository:
\url{https://github.com/Swaraj1998/axiom-beta-rfdev.git}
\end{frame}

\end{document}
