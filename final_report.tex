\documentclass{article}
\usepackage{fullpage}
\usepackage{graphicx}
\usepackage{titlepic}
\usepackage{textcomp}
\usepackage[colorlinks = true,
	    linkcolor = black,
	    urlcolor = blue,
	    citecolor = blue,
	    anchorcolor = blue]{hyperref}
\renewcommand{\baselinestretch}{2}

\title{{\Huge GSoC 2020 Final Report} \\
{\vfill\huge Linux Kernel Driver for Lattice MachXO2 programming/debugging}}
\author{Swaraj Hota (@bluez\_) \\
$<$swarajhota353@gmail.com$>$ \\
\\Mentor: Herbert P\"{o}tzl (@Bertl)}
\titlepic{\includegraphics[width=8cm]{apertus_logo.png}}
\date{\vfill\today}

\begin{document} 
\fontfamily{qag}\selectfont
\maketitle 
\newpage
\tableofcontents
\newpage

\section{Description}
The aim of this project 
(task \href{https://lab.apertus.org/T729}{\#T729})
was to make a Linux Kernel Driver to program and
debug the Lattice MachXO2 FPGAs present in AXIOM Beta. These FPGAs are used as
routing fabrics to handle all the low speed GPIO communications that are required
for plugin modules, shields and the center solder on area without sacrificing valuable
Zynq GPIOs.\newline

Two main goals of this project were:
\begin{itemize}
\item To implement an "upload`` interface to program the FPGAs
\item To implement a debug (JTAG) interface to interface OpenOCD\newline
\end{itemize}

Previously, both the MachXO2 FPGAs were programmed with python scripts which
didn't allow easy update and development.
The Linux Kernel Driver acts as a central entity to manage the FPGAs, provides
a JTAG interface to any application like OpenOCD to make SVF replays possible,
and essentially makes testing and debugging new code much easier.

\section{Work Product}
All the code produced by me during GSoC 2020 can be found in this Github repository:
\newline
\url{https://github.com/Swaraj1998/axiom-beta-rfdev}

\newpage

\section{Work Summary}

Initially, I had some idea about Linux Kernel but I never had a chance to
actually dive into the development. I found this project on GSoC and saw this
as an opportunity to finally do that. I started with the qualification task
(task \href{https://lab.apertus.org/T884}{\#T884})
and eventually started learning more about driver development.
Code can be found
\href{https://github.com/Swaraj1998/apertus-kernel-challenge}{here}.
In the \emph{\textbf{pre-community bonding period}},
I learned a lot about the hardware of AXIOM Beta and about I2C and JTAG protocols.\newline

I spent much of my time during the \emph{\textbf{community bonding period}} studying about the
I2C subsystem in Linux Kernel, and also about the whole driver subsystem in general.
I built my own kernel and setup the uboot configurations for (partial) AXIOM Betas
in the remote setup where I did all my testing. I got a basic skeleton driver
up and running in the remote betas during this period.\newline

In \emph{\textbf{Phase I}}, my milestone was to finalize the "upload`` interface in the
driver, which I was able to achieve in time. The driver was made to claim the
I2C addresses
of the "selected`` PIC (PIC West or PIC East), with the help
of a simple devicetree description. The driver was
then integrated with
\href{https://www.kernel.org/doc/html/v4.18/driver-api/fpga/fpga-mgr.html}{Linux FPGA Manager Framework},
which provides a central
sysfs interface for the driver at \emph{/sys/class/fpga\_manager/fpga\#}, and also
some callback functions to manage the programming of FPGAs. These callback
functions were then filled in with appropriate device specific implementation.
For testing, I built my own compressed bitstream using Lattice Diamond tools which
was successfully uploaded into the FPGA's SRAM. I also added some useful sysfs
attributes to read out idcode and status register bits within this period.\newline

In \emph{\textbf{Phase II}}, my milestone was to figure out a debug interface for OpenOCD
and implement at least some debug functionalities, which I was not able to meet
within this phase due to some unexpected issues and some underestimation on my part.
I spent much of this phase improving various aspects of the driver. More
useful sysfs attributes, more abstract functions, and printing proper debug
info where among them.
The important part (which I got stuck on) was to make the driver work with
both the PICs (and hence both the FPGAs) by \emph{seamlessly switching} between them.
This switching (or "selecting``, as mentioned before) was previously done with
a python script (rf\_sel.py), and hence the driver could only work with one FPGA
at a time, whichever was selected before loading the driver.
Now, doing this \emph{seamless switching} turned out to be more difficult than it looked,
but it was
nonetheless done with less changes on the driver part but more on the
devicetree part.\newline

Depending on the power board version of the AXIOM Beta, there were two different
muxes (PCA9540 \& PCA9543) and another analog switch (TS3A4751) to work with. 
For the muxes, there was an existing Linux Kernel Driver (i2c-mux-pca954x) which
was used through appropriate devicetree entries. This driver created two new
multiplexed I2C buses which my driver could use to talk to each PIC. For the
analog switch also there was an existing Linux Kernel Driver (i2c-mux-gpio) which
was used similarly (albeit through GPIO expander present at I2C bus 0 in AXIOM
Beta). All these were eventually implemented though I could not complete it by
the end of this phase. The issue turned out to be an error in the devicetree
entry which, due to lack of proper documentation, was very difficult to find and fix.
But in the way I did learn a lot about devicetree and Linux Kernel debugging.\newline

Finally, in \emph{\textbf{Phase III}}, I was able to fix the issue regarding the
\emph{seamless switching},
after lots and lots of debugging. I provided a devicetree overlay
for each of the muxes/switch, the appropriate one of which must be loaded before
loading the driver. The driver can then automatically take care of which
FPGA to talk to. Two different FPGA managers are registered (one for each FPGA),
having separate entries under \emph{/sys/class/fpga\_manager/}. Once this
was taken care of I could finally focus on the debug interface of the driver.
We decided to go for an ioctl interface very similar to one mentioned in a
new (currently unpatched)
\href{https://lore.kernel.org/patchwork/cover/1223367/}{Linux Kernel Driver patchwork for JTAG support}.
The interface could be accessed via \emph{/dev/rfw} and \emph{/dev/rfe} for
routing fabric west and east respectively.\newline

After I had the ioctl interface finalized (and tested with sample application code),
I started working on OpenOCD codebase to write a debug adapter for it.
In the very last week, after some rounds of testing and debugging,
I was finally able to produce a patch for OpenOCD that worked with my driver.
I tested it with SVF replays to flash each of the FPGAs, which was successfully
executed by OpenOCD using my driver.\newline

\section{Current Status and Future Work}

The initial goal of the project, as mentioned in GSoC 2020 Task 
\href{https://lab.apertus.org/T729}{\#T729}, is successfully completed.
The driver can currently upload a compressed bitstream produced by Lattice Diamond
tools through the FPGA Manager Framework (using "firmware`` attribute).
Some useful sysfs attributes like
idcode, stat, statstr, digest, traceid, and usercode are added (details can
be found in README.md of the Github repository). Using the OpenOCD patch provided
in the repository, OpenOCD can be used to run SVF replays on the MachXO2 FPGAs
through the driver's ioctl interface.\newline

Even though the initial goal is met, there is still a lot to improve in terms
of code design, performance and code quality. For instance, the driver is
"hardcoded`` in many ways to work with just RFWest and RFEast. As my mentor
suggests, more MachXO2 FGPAs could also be added in the setup in future and hence
it should be made generic. Also as it does not make sense for the driver to know
weather the FPGA is "West`` or "East``, just referencing them through their
device instance will be a good idea. Further, the reads and writes to the PIC
can also be optimized in many ways. I will be working after GSoC on all
these and finally deploy a polished driver in-tree with the AXIOM Beta Kernel.

\newpage

\section{Challenges faced}

Working on this project was quite challenging, and yet rewarding. Many challenges
I faced were based on lack of documentation or explanation of things, especially
when working with devicetrees and MachXO2 FPGA. Also, writing kernel code is
very critical, as the system could break if anything goes wrong. Keeping that
in mind, I had to think
a lot before writing even a single line of code (and even that wasn't enough).
I reused well tested code wherever possible. Debugging the kernel,
and specifically the devicetree, was also quite challenging as there is no
good upstream support.
I had little time to understand and implement the OpenOCD side of things,
as it was an unknown codebase for me, and I haven't really
ever had used it before.\newline

All in all, the learning curve was quite steep, as there were so many
things involved, from Python scripts to Kernel code to OpenOCD code, and there
also was hardware (and sometimes its weird issues). Nevertheless, my mentor
was always helpful and ever available for any issues I faced. We also debugged
a lot of issues together (@IRC) and I learned a lot from him in the process.

\section{In Conclusion...}

Initially, my perception of the project was that I would learn about Linux
Kernel Development and in general about hardware stuff.
I eventually ended up getting (at least some)
exposure to VHDL, Lattice Diamond tools, FPGAs, MCUs, Muxes/Switches,
Python, TCL, Makefiles,
Bash, Kernel configuration and build process, devicetree (and overlays), protocols like JTAG \& I2C,
reading PCB schematics, OpenOCD, and \LaTeX, among other things.
My mentor always managed to push me beyond my comfort zone, and I think that is
why I got exposure to so many things.
I would like to
thank him and other developers who helped me in various ways throughout the program.
Working with {apertus\textdegree} Association this year at GSoC was really fun,
and I would like to continue to be a part this organization even after GSoC ends. 

\end{document}
