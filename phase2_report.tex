\documentclass{article}
\usepackage{fullpage}
\usepackage{graphicx}
\usepackage{titlepic}
\usepackage{textcomp}
\usepackage{hyperref}
\renewcommand{\baselinestretch}{2}

\title{{\Huge GSoC Phase-II Report} \\
{\vfill\huge Linux Kernel Driver for Lattice MachXO2 programming/debugging}}
\author{Swaraj Hota (@bluez\_) \\
\\Mentor: Herbert Poetzl (@Bertl)}
\titlepic{\includegraphics[width=8cm]{apertus_logo.png}}
\date{\vfill\today}

\begin{document} 
\fontfamily{qag}\selectfont
\maketitle 
\newpage
\tableofcontents
\newpage

\section{Introduction}
The aim of this task (T729) was to make a Linux kernel driver to program/debug
MachXO2 FPGAs (used as routing fabrics) in AXIOM Beta main board. This is a
report of my progress in
the \emph{second} phase of GSoC 2020.\newline
Two main goals of this project are:
\begin{itemize}
\item To implement an "upload`` interface to program MachXO2
\item To implement a debug interface for OpenOCD
\end{itemize}

\section{Progress}
By the end of the \emph{first} phase I was able to successfully implement an
upload interface to program the routing fabrics, along with sysfs attributes to
read out idcode and status register bits. 
My milestone for the \emph{second} phase (as I mentioned in my proposal) was to
figure out a debug interface for OpenOCD and implement at least some debug
functionalities by the end of the phase.
I, unfortunately, was not able to meet this milestone due to some issues that
came up and some underestimation on my part.\newline

I spent most of this phase improving various aspects of the driver, added more
useful sysfs entries, and spent much time tackling with issues that came up while 
implementing the mux switching between the two PICs using an existing kernel
driver (pca954x). Devicetree entries are used to bind the mux driver as well as
our rfdev driver. Although the mux driver binds successfully and creates two
multiplexed buses for the two PICs, our rfdev driver could not be registered
with those buses due to some reason which is still being debugged. Besides that
I did a lot of reading on OpenOCD and a possible implementation for the debug
interface. Herbert and I had a discussion with people on OpenOCD IRC, they suggested
us to look into their current USB driver on OpenOCD side and also a proposed
patchset for linux kernel about an ioctl based implementation 
\href{https://lore.kernel.org/patchwork/cover/874787/}{[link]}. So in the next
phase I will be picking it up from there.\newline

Some important issues that needed to be addressed before making the debug
interface where taken care of. A way to reset the FPGAs through JTAG,
parsing of status register, more abstract functions, two different FPGA manager
instances, and sysfs entries like traceid, usercode, digest, statstr which will
be useful for debugging, were all implemented. As we have different types of muxes/switches
across different power board versions, we will need to address each of these
in a proper way in the next phase as quickly as possible, before moving on to
the most important part of the project i.e. the debug interface.


\section{Phase Timeline}

\paragraph{In this phase:}
\begin{itemize}
\item \emph{Week 1:} 
	\begin{itemize}
	\item Investigated a way to optimize the byte reverse function (did not find a suitable way) 
	\item Added sysfs entry to show md5 digest of the latest uploaded firmware 
	\item Read about OpenOCD and explored a possible debug interface
	\end{itemize}
\item \emph{Week 2:} 
	\begin{itemize}
	\item Added code to parse status register bits, along with debug messages
		and a sysfs entry for human readable status (statstr)
	\item Added basic operational state reporting through sysfs entry 'state'
		(needs to be improved)
	\item Added two separate FPGA managers (currently both work for same FPGA) 
	\end{itemize}
\newpage
\item \emph{Week 3 \& 4:} 
	\begin{itemize}
	\item Added function to reset FPGA through JTAG (currently resets on driver exit)
	\item Added sysfs entries to read out 'traceid' (unique MachXO2 id) and 'usercode'
	\item Cleaned up code and added more abstract functions like JTAG command in and command out
	\item Tried to add mux switching functionality through devicetree overlays (still ongoing)
	\end{itemize}
\end{itemize}


\section{Challenges Faced}
This phase was quite challenging as I had to work with things that I was very
unfamiliar with. First was to properly understand how OpenOCD works and how to
go about making a debug interface. After a lot of reading and discussion with OpenOCD
people I was finally able to imagine a possible implementation. The second big
challenge was working with devicetree to properly load the mux driver along
with our rfdev driver. This task is still ongoing and I have been stuck at it since
more than a week. Herbert and I have been trying to debug what the issue could be,
as of yet we suspect it could also be some bug in linux i2c subsystem, as everything
seems to be in place and it should theoretically work but it just doesn't. Trying
to find a suitable way to debug the devicetree was also a challenge as there is
no good support and very little up-to-date documentation.

\section{Modified Timeline}
Major task in the next phase would also be to finally implement a debug interface
for OpenOCD as I was not able to do it in this phase. Before that, the muxes/switches
also needs to be taken care of.

\newpage

\paragraph{In the upcoming phase:}
\begin{itemize}
\item \emph{Week 1}:
	\begin{itemize}
	\item Fix the mux related issue where the driver does not get probed
	\item Add support for the different
		muxes/switches across different power boards we have
	\end{itemize}
\item \emph{Week 2}:
	\begin{itemize}
	\item Change the PIC firmware to allow configuration of its JTAG port and
		possibly also reset the FPGAs, directly in response to a command byte
	\item Finalize idea of the debug interface implementation
		(possibly the ioctl interface mentioned previously)
	\end{itemize}
\item \emph{Week 3 \& 4}: 
	\begin{itemize}
	\item Implement the debug interface for OpenOCD
	\item Take care of other deferred work, clean up the code
	\end{itemize}
\end{itemize}

\section{In Conclusion...}
The driver was improved in various aspects in this phase. I have been trying my
best to keep up with the steep learning curve. I got stuck with unexpected issues
yet again, but learned a lot along the way as well. Next phase is going to
be even more challenging and I am not going to underestimate that this time.
Nevertheless, even if for some reason I could not fully complete the project by
the end of next phase, I will definitely work on it even after GSoC ends. I'll
make sure the driver is fully usable and helps to accelerate the development process
of routing fabrics.

\end{document}
